\documentclass[11pt]{article}
% Packages
\usepackage[utf8]{inputenc}
\usepackage{amsmath,amssymb}
\usepackage{graphicx}
\usepackage{hyperref}
\usepackage{listings}
\usepackage{color}
\usepackage{algorithm}
\usepackage{algorithmic}
\usepackage{geometry}
\geometry{margin=1in}

% Title
\title{LLMスコアリングによる軽量レコメンドエンジン\\早期停止と公平性を考慮した候補抽出}
\author{匿名投稿}
\date{\today}

\begin{document}

\maketitle

\begin{abstract}
LLMを用いて候補を評価する軽量なレコメンドアーキテクチャを提案する。特徴量設計や事前学習を必要とせず、ユーザーと候補のペアをプロンプト経由でスコアリングする。一定のスコアを超える候補が\(k\)件見つかれば処理を打ち切る早期停止により、応答時間とコストを削減する。さらにロングテール排除を防ぐため、低スコア候補をランダムに挿入する。本稿では仕組みと活用例を概説する。
\end{abstract}

\section{はじめに}
近年、オンラインサービスにおいて推薦エンジンは欠かせない存在である。従来手法では複雑な特徴量設計や継続的な学習が必要だったが、汎用的なLLMの登場によりプロンプトベースの新しいアプローチが可能となった。

\section{アーキテクチャ概要}
本システムは以下の構成要素から成る。
\begin{itemize}
    \item 入力: 対象ユーザーと候補アイテムの集合
    \item 前処理: ビジネスメトリクス (例: 購入金額) で候補をソート
    \item スコアリング: 各ユーザーと候補の組をLLMで並列評価
    \item 早期停止: \(k\)件が閾値\(T\)を超えた時点で評価を停止
    \item ランダム挿入: 露出バイアスを避けるため低スコア候補を確率的に混入
\end{itemize}

\begin{figure}[h!]
    \centering
    \includegraphics[width=0.85\textwidth]{architecture_diagram.png} % placeholder
    \caption{LLMスコアリングと早期停止を備えたシステム構成}
\end{figure}

\section{LLMによるスコアリング}
各ユーザーと候補のペアは次のようなプロンプトで評価される。
\begin{quote}
"ユーザーAは30代のデータサイエンティストで実務的。候補Bはデザイン出身のプロダクトマネージャーで情熱的かつ行動力がある。二人が新規製品の立ち上げでどの程度協働できるか、0--10で評価し理由を述べよ。"
\end{quote}
LLMは数値スコアと簡潔な理由を返し、これをランキングに利用する。

\section{早期停止メカニズム}
すべての候補を評価する代わりに、閾値\(T\) (例: 7.5) と結果数\(k\) (例: 3) を設定し、条件を満たした時点で残りの評価をスキップする。これにより大規模な候補集合でもコストとレイテンシを大幅に削減できる。

\section{公平性を考慮した候補挿入}
活動量の少ないユーザーや低支払いユーザーが埋もれないよう、上位\(N\)件中の一部にランダムで低ランク候補を挿入する。これにより多様性を保ち、長期的な公平性を向上させる。

\section{考察と応用例}
本アーキテクチャは次のような場面で利用できる。
\begin{itemize}
    \item タレントやメンターのマッチングサービス
    \item コラボレーターやアドバイザーの推薦
    \item パーソナライズされたコンテンツ配信や紹介
\end{itemize}
従来の重い機械学習パイプラインとは異なり、ゼロショットのLLM呼び出しのみで動作し、目的に応じてプロンプトを変更するだけで柔軟に運用できる。

\section{まとめ}
本稿では、プロンプトによるスコアリング、早期停止、ランダム挿入を組み合わせたLLMベースの軽量レコメンド手法を示した。LLMネイティブ環境で実用的な推薦システムを少ない工数で構築できることを示唆する。

\end{document}
